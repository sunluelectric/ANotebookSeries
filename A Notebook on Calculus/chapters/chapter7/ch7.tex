\chapter{Multiple Integral} \label{ch7ch}

The integral for scalar input function has been introduced in Chapter \ref{ch3}. The integral of $y=f(x)$ with the lower bound $a$ and upper bound $b$ is denoted by \eqref{ch3eq:generaldefiniteintegral2}. It is defined as the limit of sum given by \eqref{ch3eq:generaldefiniteintegral}, and can be interpreted as the area circulated by $x=a$, $x=b$, $y=f(x)$ and $y=0$ as shown by Fig. \ref{ch3fig:explainintegrial}. In practice, the integral can be calculated using \eqref{ch3eq:calculatedefiniteintegral}.

In this chapter, the integral for multiple input functions is introduced. A motivating example is used to illustrate the basic concept and meaning of multiple integral in Section \ref{ch7sec:motivatingexp}, and its formal definition is given in Section \ref{ch7sec:multipleintegral}.

\section{A Motivating Example} \label{ch7sec:motivatingexp}

\section{Multiple Integral} \label{ch7sec:multipleintegral}

