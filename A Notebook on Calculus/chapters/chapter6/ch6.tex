\chapter{Partial derivative} \label{ch6ch}

Partial derivative studies the effect of a small deviation of one particular independent variable on the multivariable function. It is similar with the normal derivative in many ways, but also has its unique characteristics.

In Section \ref{ch6sec:motivatingexp}, a motivating example is given to illustrate the motivation of introducing partial derivative. In Section \ref{ch6sec:partialtotalderivative}, the definition of partial derivative is given. In Sections \ref{ch6sec:gradient} and \ref{ch6sec:jacobianmatrix}, two very important and commonly used tool derived from partial derivative, namely gradient and Jacobian matrix, are introduced respectively.

\section{A Motivating Example} \label{ch6sec:motivatingexp}

Consider the following motivating example where $y=f(x_1, x_2)$ is a multivariable function with $2$ inputs.

\begin{shortbox}
\Boxhead{A Motivating Example}

Consider
\begin{eqnarray}
    y &=& 2x_1^2 + 5x_2^2 -2x_1x_2 - 6x_1 - 6x_2 + 9. \label{ch6eq:motivatingexample}
\end{eqnarray}

Q1: Let $x_2 = 1$ be a constant. Derive $y$ as a function of $x_1$, and calculate its derivative with respect to $x_1$. Similarly, let $x_1 = 1$ be a constant and derive $y$ as a function of $x_2$, and calculate its derivative with respect to $x_2$.

Q2: At $(x_1, x_2) = (1, 1)$, consider small vibrations $\Delta x_1$ and $\Delta x_2$. Describe $\Delta y$ as a function of $\Delta x_1$ and $\Delta x_2$.

Q3: Find such $x_1$ and $x_2$ that $y$ is minimized.

\end{shortbox}

Equation \eqref{ch6eq:motivatingexample} can be plot in 3-D as follows.



\section{Partial and Total Derivative} \label{ch6sec:partialtotalderivative}

\section{Gradient} \label{ch6sec:gradient}

\section{Jacobian Matrix} \label{ch6sec:jacobianmatrix}
