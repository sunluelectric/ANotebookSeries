\chapter*{Foreword}
If a piece of software or an e-book can be made completely open source, why not a notebook?

This brings me back to the summer of year 2009, when I just started my third year as a high school student in Harbin No. 3 High School. In around August and September of every year, that is, when the results of Gaokao (National College Entrance Examination of China, annually held in July) are released, people from photocopy shops will start selling notebooks photocopies that they claim to be of the top scorers of the exam. Much as I was curious about what these notebooks look like, I myself did not expect to actually learn anything from them, mainly for the following three reasons.

First of all, some (in fact many) of these notebooks were more tough to understand than the textbooks. I guess we cannot blame the top scorers for being too smart and making things sometimes extremely brief or overwhelmingly complicated.

Secondly, why would I wanted to adapt to notebooks of others when I had my own, which should be as good as theirs.

And lastly, as a student in the top high school myself, I knew that the top scorers of the coming year would probably be a schoolmate or a classmate. Why would I want to pay that much money to a complete stranger in a photocopy shop for my friend's notebook, rather than asked from him or her directly?

However, things had changed after my becoming an undergraduate student in year 2010. Since in the university there were so many modules and materials to learn, students were often distracted from digging into one book or module very deeply. (For those who were still able to do so, you have my highest respect.) The situation got even worse as I became a Ph.D. student in year 2014, this time due to that I had to focus on one research topic entirely, and could hardly split much time on other irrelevant but still important and interesting contents.

This motivated me to start reading and taking notebooks for selected books and articles such as journal papers and magazines, just to force myself to spent time learning new subjects. I usually used hand-written notebooks. My very first notebook was on \textit{Numerical Analysis}, an entrance level module for engineering background graduate students. Till today I have on my hand dozens of notebooks, and one day it suddenly came to me: why not digitalize them, and make them accessible online and open source, and let everyone read and edit it?

\noindent ---

\noindent As majority of open source software, this notebook (and it applies to the other notebooks in this series) does not come with any ``warranty'' of any kind, meaning that there is no guarantee for the statement and knowledge in this notebook to be exactly correct as it is not peer reviewed. \textbf{Do NOT cite this notebook in your academic research paper or book!} Of course, if you find anything here useful with your research, feel free to trace back to the origin of the citation, and double confirm it yourself then on top of that determine whether or not to use it in your research.

This notebook is suitable as:
\begin{itemize}
  \item a quick reference guide;
  \item a brief introduction for beginners of the module;
  \item a ``cheat sheet'' for students to prepare for the exam (Don't bring it to the exam unless it is allowed by your lecture!) or for lectures to prepare the teaching materials.
\end{itemize}

This notebook is NOT suitable as:
\begin{itemize}
  \item a direct research reference;
  \item a replacement to the textbook;
\end{itemize}
because as explained the notebook is NOT peer reviewed and it is meant to be simple and easy to read. It is not necessary brief, but all the tedious explanation and derivation, if any, shall be ``fold into appendix'' and a reader can easily skip those things without any interruption to the reading.

\noindent ---

Although this notebook is open source, the reference materials of this notebook, including textbooks, journal papers, conference proceedings, etc., may not be open source. Very likely many of these reference materials are licensed or copyrighted. Please legitimately access these materials and properly use them if necessary.

Some of the figures in this notebook is drawn using Excalidraw, a very interesting tool for machine to emulate hand-writing. The Excalidraw project can be found in GitHub, \textit{excalidraw/excalidraw}.
