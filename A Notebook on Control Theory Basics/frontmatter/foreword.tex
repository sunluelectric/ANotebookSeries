\chapter*{Foreword}
If a piece of software or an e-book can be made completely open source, why not a notebook?

This brings me back to the summer of year 2009, when I just started my third year as a high school student in Harbin No. 3 High School. In around August and September of every year, that is, when the results of Gaokao (National College Entrance Examination of P. R. China, annually held in July) are released, you would find people selling notebooks photocopies claimed to be collected from the top scorers of the exam. Much as I was interested in what these notebooks look like, I myself was not expecting to actually learn anything from them, mainly for the following three reasons.

First of all, some (in fact many) of these notebooks were more difficult to understand than the textbooks. I guess we cannot blame the top scorers for being too smart and make things sometimes extremely brief, or otherwise overwhelmingly complicated.

Secondly, why would I want to adapt to notebooks of others when I have my own? And by the way, I was positive that mine would be as good as theirs, given that I had been putting the same time (three years of high school, only for 6 modules!) and effort learning the courses and preparing the notebooks.

And lastly, as a student in Harbin No. 3 High School, I knew that the top scorers of the coming year would probably be a schoolmate next door, perhaps even a good friend of mine. Why would I want to pay a great amount of penny to a complete stranger in a photocopy shop for his or her notebook, rather than ask from him or her directly?

However, things have changed later on after entering a university as an undergraduate student. I think the main cause of the change is that, since in the university there are so many modules and materials to learn, students are often distracted from digging into one book or module very deeply. (For those who still can concentrate, you have my highest respect.) The situation becomes even worse as I become a Ph.D. student, this time due to that I have to concentrate on one subject entirely, and can hardly split much time on other irrelevant but still important and interesting contents.

This motivates me to start reading and taking notebooks for selected books and articles such as journal papers and magazines. I have a bunch of notebooks with me, most of them are physical. My very first notebook is on \textit{Numerical Analysis}, an entrance level module for engineering background students. Till today I have on my hand dozens of notebooks. One day it suddenly came to me: why not digitalize them, and make them accessible online and open source, and let everyone read and edit it?

\noindent ---

\noindent As majority of open source software, this notebook (and it applies to the other notebooks in this series) does not come with any ``warranty'' of any kind, meaning that there is no guarantee for the statement and knowledge in this notebook to be exactly correct as it is not peer reviewed. \textbf{Do NOT cite this notebook in your academic research paper or book!} Of course, if you find anything here useful with your research, please trace back to the origin of the citation, and read it yourself, and on top of that determine whether or not to use it in your research.

This notebook is suitable as:
\begin{itemize}
  \item a quick reference guide;
  \item a brief introduction to the subject;
  \item a ``cheat sheet'' for students to prepare for the exam (Don't bring it to the exam unless it is allowed by your lecture!) or for lectures to prepare the teaching materials.
\end{itemize}

This notebook is NOT suitable as:
\begin{itemize}
  \item a direct research reference;
  \item a replacement to the textbook;
\end{itemize}
because as explained the notebook is NOT peer reviewed and it is meant to be simple and easy to read. It is not necessary brief, but all the tedious explanation and derivation, if any, shall be ``fold into appendix'' and a reader can easily skip those things without any interruption to the reading.

\noindent ---

\noindent Although this notebook is open source, the reference materials of this notebook, including many textbooks, journal papers, conference proceedings, etc., may not be open source. Very likely many of these reference materials are licensed or copyrighted. Please legitimately access these materials and properly use them if necessary. \vadjust{\vfill\pagebreak}
