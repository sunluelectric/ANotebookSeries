%\chapterauthor{Author Name}{Author Affiliation}
%\chapterauthor{Second Author}{Second Author Affiliation}
\chapter{Shell}

Linux command line tool, usually known as the ``shell'', is the most powerful tool for Linux operations including configuration and control of the OS as well as other hardware/software of the system. Notice that the use of shell is not compulsory for casual users when the graphical desktop is present. Though the shell is not as intuitive as the graphical tools, the shell is more powerful and flexible, and well-supported by the community.

Linux shell will be used repeatedly in the remaining sections of this notebook for different functions.

\section{Brief Introduction}

Linux command line tool, usually known as Linux shell, was invented before the graphical tools, and it has been more powerful and flexible than the graphical tools from the first day. On those machines where no graphical desktops are installed, the use of shell is critical.

There are different types of shells. The most commonly used shell is the ``bash shell'' which stands for ``Bourne Again Shell'', derived from the ``Bourne Shell'' used in UNIX.

Some other shells such as ``C Shell'' and ``Korn Shell'' are also popular among certain users or certain Linux distributions. In case where your Linux distribution does not have these shells pre-installed, you can install and use these shells just like installing other software.

In this notebook, we will mostly focus on the bash shell.

\section{Basic Concepts}

After opening the shell or terminal, you will see a string (usually containing username, hostname, current working directory, etc.) followed by either a \verb|$| or \verb|#|, starting from where you can input your shell command. For example, it may look like the following:
\begin{verbatim}
username@hostname:~$
\end{verbatim}

The above displayed string is called a \textit{prompt}, indicating the start of a manually input command. By default, for regular user, the ending of the prompt is \verb|$| while for the root user, the ending is \verb|#|.

By saying root user, we are referring to a special user whose username and user ID (UID) ``root'' and $0$ respectively. This UID gives him the administration privilege over the machine, such as adding/removing users, change ownership of files etc. To avoid significant damage by human error, root user shall not be used unless it is definitely necessary. For this reason, in many servers the root user is deactivated (for example, by setting its login password to invalid).

Notice that a root user is different from regular user who is equipped with \textit{sudo privilege}, though a regular user with sudo privilege can temporarily switch to root user by using \verb|su| as follows.
\begin{verbatim}
regularuser@hostname:~$ sudo su
[sudo] password for regularuser:
root@hostname:/home/regularuser#
\end{verbatim}

More about sudo privilege, \verb|sudo| and \verb|su| commands are introduced later parts of the notebook.

You can key in a command after the prompt, and execute the command by pressing the \verb|Enter| key. A Linux shell command usually has the following form.
\begin{verbatim}
$ <command> <configuration-arguments> <input>
\end{verbatim}


\section{Useful Basic Commands}

Selected useful commands are categorized as follows.

\vspace{0.1in}
\noindent \textbf{Show System Information}
\vspace{0.1in}

Administrative users may need to frequently check the basic system information, such as hardware configuration, OS version, username, hostname, disk usage, running process, system clock, etc. Some useful commands are summarized below.

The following command \verb|lshw| lists down hardware information in details. Sudo privilege is recommended when using this command, to give detailed and accurate information of the system. Since the displayed information is so detailed and can take up many screens, sometimes it is more convenient to use \verb|-short| argument.
\begin{verbatim}
$ sudo lshw
\end{verbatim}



\vspace{0.1in}
\noindent \textbf{Navigate Files and Folders}
\vspace{0.1in}



\vspace{0.1in}
\noindent \textbf{Alias and Shortcuts}
\vspace{0.1in}

\section{Shell Environment Configuration}


\section{Shell Script Programming}













