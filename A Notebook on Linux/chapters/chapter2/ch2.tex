%\chapterauthor{Author Name}{Author Affiliation}
%\chapterauthor{Second Author}{Second Author Affiliation}
\chapter{Shells}

Linux command line tool, usually known as the ``shell'', is the most powerful tool for Linux operations including configuration and control of the OS and application software. Though not as intuitive as the graphical tools, the shell is more powerful and flexible, and well-supported by the community.

Linux shell will be used repeatedly in the remaining sections of this notebook.

\section{Brief Introduction}

Linux command line tool, usually known as Linux shell, was invented before the graphical tools became available, and it has been more powerful and flexible than the graphical tools from the first day. On those machines where no graphical desktops are installed, the use of shell is critical.

There are different versions of shells. The most commonly used shell is the ``bash shell'' which stands for ``Bourne Again Shell'', derived from the ``Bourne Shell'' used in UNIX.

Some other shells such as ``C Shell'' and ``Korn Shell'' are also favored among certain users or certain Linux distributions. In this notebook, however, we will mostly focus on only the bash shell.

\section{Basic Concepts}

After opening the shell or terminal, you will see a string (usually containing username, hostname, current working directory, etc.) followed by either a \verb|$| or \verb|#|, starting from where you can input your shell command. For example, it may look like the following:
\begin{verbatim}
username@hostname:~$
\end{verbatim}

The \verb|$| and \verb|#| is called a \textit{prompt}, indicating the start of a manually input command. By default, for regular user, the prompt is \verb|$| while for the root user, the prompt is \verb|#|. 

By saying root user, we are referring to a special user whose username is ``root'' by default. This user has a user ID (UID) of $0$, which gives him the unlimited administration privilege over the machine. To avoid possible mis-operation by human error, root user shall not be used unless it is definitely necessary. For this reason, in many applications the root user is deactivated (for example, by setting its login password to invalid).

Notice that root user is different from regular user with \textit{sudo privilege}, though a regular user with sudo privilege can temporarily switch to root user by using \verb|su| as follows.
\begin{verbatim}
regularuser@hostname:~$ sudo su
[sudo] password for regularuser:
root@hostname:/home/regularuser#
\end{verbatim}

More about sudo privilege, \verb|sudo| and \verb|su| commands are introduced later parts of the notebook.

\section{Useful Commands}

















