%\chapterauthor{Author Name}{Author Affiliation}
%\chapterauthor{Second Author}{Second Author Affiliation}
\chapter{General Introduction to Linux}

This chapter gives a brief introduction Linux, including some of its key features and advantages/disadvantages over other operating systems.

\section{Brief Introduction to Linux}

Linux is an operating system. An operating system is essentially a special piece of software running on a computer (or servers, or mobile devices, or any other electrical device that is complicated enough and requires an operating system) that manages hardware resources of the system and provide services to the upper application layer software. To be more precise, an operating system shall be able to
\begin{itemize}
  \item detect and prepare hardware;
  \item manage processes;
  \item manage memory;
  \item provide user interface and user authentication;
  \item manage file systems;
  \item provide programming tools for creating applications.
\end{itemize}

Linux has been overwhelmingly successful and has been adopted in many different areas. For example, Android operating system for mobile phones is developed using Linux. Google Chrome is also backed by a Linux operating system. Many famous websites including Facebook are also running on Linux servers.

Some of the most favorable features of Linux (especially to large size enterprises) are as follows.
\begin{itemize}
  \item Clustering: multiple sets of hardware appearing to be one computer.
  \item Visualization: different hosts running on the same hardware set, each of which appears to be on a separate computer with specialized features.
  \item Cloud computing: flexible resources management realized by running applications on cloud on virtual Linux computer.
  \item Real-time computing: embedded Linux based microcontrollers or computers.
\end{itemize}

Linux differs from Microsoft Windows and MacOS in many ways, though they are all very good operating systems. Among the three operating systems, Linux is the only one that is completely open source (in the sense that all its code can be viewed and modified per requested), thus adding a lot of flexibility when using it.

\section{A Short History of Linux}

The initial motivation of Linux is to create a UNIX-like operating system that can be freely distributed in the community.

Many modern computer systems including MacOS and Linux are derived from UNIX. UNIX operating system was created by AT\&T in 1969 as a better software development environment that AT\&T used internally. In 1973, UNIX was rewritten in C language, thus adding more useful features such as portability. Today, C is still the primary language used to create UNIX (and also Linux) kernels.

AT\&T, who originally owned UNIX, tried to make UNIX into a commercial product. However, back then AT\&T was restricted from selling computers. Therefore, AT\&T decided to license UNIX source code to universities for a nominal fee. Researchers from universities start learning and improving UNIX, which speed up the development of UNIX. In 1976, UNIX V6 became the first UNIX that was widely spread. UNIX V6 is developed at UC Berkeley and was named the Berkeley Software Distribution (BSD). From then on, UNIX moved towards two separate directions: BSD continued forward in the ``open'' and ``share'' manner, while AT\&T started steering UNIX toward commercialization and by 1984 AT\&T was pretty ready to start selling UNIX (i.e. the commercialized version is mostly famous by the name ``AT\&T: UNIX System Laboratories (USL)''). USL did not sell very well. As said, AT\&T could only sell source but not complete boxes of PC, and for this reason the price for the source code had to be higher than other OS (such as Microsoft Windows). Other companies, such as SCO and Sun Microsystems, were more successful sellers by selling UNIX based PC and workstations for high-end users. Overall, UNIX source code was extremely expensive.

In 1984, Richard Stallman started the GNU project as part of the Free Software Foundation. I is recursively named by phrase ``GNU is Not UNIX'', intended to become a recording of entire UNIX that could be open and freely distributed. The community started to ``recreate'' UNIX based on the defined interface protocols published by AT\&T. The BSD got a good chance initially, it eventually failed due to the doubts people held for open source codes.

Linus Trovalds started creating his version of UNIX, i.e. Linux, in 1991. He managed to publish the first version of the Linux kernel on August 25, 1991, initially only worked for 386 processor. Later in October, Linux 0.0.2 was released with many parts of the code rewritten in C language, making it more suitable for cross-platform usage. This Linux kernel was the last and the most important piece of code to complete a UNIX-like system under GPL. It is so important that people call this operating system ``Linux OS'' instead of ``GNU OS'', although is the host of the project and Linux kernel is only a (most important) part of it.

As casual Linux users, people do not want to understand and compile the Linux source code to use Linux. To respond to this need, different Linux distributions have merged. They share the same OS kernel, but present themselves differently in many ways, for example in the way they manage software applications or design user interfaces. Today, there are hundreds of Linux distributions in the community. Two famous distributions in the community (and their variations) are listed below.
\begin{itemize}
  \item Red Hat Distribution
  \begin{itemize}
    \item Red Hat Enterprise Linux (RHEL)
    \item Fedora
    \item CentOS
  \end{itemize}
  \item Debian Distribution
  \begin{itemize}
    \item Ubuntu
    \item Linux Mint
    \item Elementary OS
    \item Raspberry Pi OS
  \end{itemize}
\end{itemize}






















