%\chapterauthor{Author Name}{Author Affiliation}
%\chapterauthor{Second Author}{Second Author Affiliation}
\chapter{General Introduction to Linux}

This chapter gives a brief introduction Linux, including some of its key features and advantages/disadvantages over other operating systems.

\section{Brief Introduction to Linux}

Linux is an operating system. An operating system is essentially a special piece of software running on a computer (or servers, or mobile devices, or any other electrical device that is complicated enough and requires an operating system) that manages hardware resources of the system and provide services to the upper application layer software. To be more precise, an operating system shall be able to
\begin{itemize}
  \item detect and preparing hardware
  \item manage processes
  \item manage memory
  \item Proving user interface and user authentication
  \item Managing file systems
  \item Provide programming tools for creating applications
\end{itemize}

Linux has been overwhelmingly successful and has been adopted in many different areas. For example, Android operating system for mobile phone is developed on top of Linux. Google Chrome is developed from Linux. Websites such as Facebook are also developed using Linux as its base. Many cloud services existing now are also running on Linux on remote servers. A quick glance of the advantages/unique features of Linux over other systems are as follows (they are useful when comes to large enterprises and large size servers).
\begin{itemize}
  \item Clustering: multiple sets of hardware appearing to be one computer
  \item Visualization: different hosts running on the same hardware set, each of which appears to be on a separate computer with specialized features
  \item Cloud computing: flexible resources management realized by running applications on cloud on virtual Linux computer
  \item Real-time computing: embedded Linux based micro controller or computer
\end{itemize}

\section{Key Features of Linux}

Linux differ from Microsoft Windows and MacOS in many different ways, though they are all very good operating systems. Among the three operating systems, Linux is the only one that is completely open source (in the sense that all its code can be checked), thus is the most flexible operating system.

\subsection{Filesystem Hierarchy}

\subsection{Access Control Lists}




